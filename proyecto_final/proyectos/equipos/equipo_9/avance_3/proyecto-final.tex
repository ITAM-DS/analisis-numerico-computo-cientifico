% !TeX spellcheck = es_MX
\documentclass[12pt,reqno,letter]{article}
\usepackage[spanish, mexico]{babel}
\usepackage[utf8]{inputenc}
\usepackage[T1]{fontenc}
\usepackage{graphicx} %allows you to use jpg or png images. PDF is still recommended
\usepackage[colorlinks=False]{hyperref} % add links inside PDF files
\usepackage{amsmath}  % Math fonts
\usepackage{amsfonts} %
\usepackage{amssymb}  %
\usepackage{multicol}
\usepackage{hyperref} % agrega links
%% Sets page size and margins. You can edit this to your liking
\usepackage[top=1.3cm, bottom=2.0cm, outer=2.5cm, inner=2.5cm, heightrounded,
marginparwidth=1.5cm, marginparsep=0.4cm, margin=2.5cm]{geometry}
%% \usepackage[authoryear]{natbib}
\usepackage[affil-it]{authblk}
\bibliographystyle{abbrvnat}
\usepackage{enumitem}
%\PrerenderUnicode{ü}

\begin{document}
	\title{ Implementación del Método de Longstaff \& Schwartz para valuación de opciones americanas }
	\author{Jorge III Altamirano Astorga - 175904, Eduardo Selim Martínez Mayorga - 175921, Ariel Ernesto Vallarino Maritorena - 175875}
	\affil{Instituto Tecnológico Autónomo de México}
	\date{Mayo, 2019}
	\maketitle
	
	\begin{abstract}
		Proyecto donde utilizamos la técnica aplicada en la investigación de Longstaff-Schwartz (2001, UCLA) denominada  “Aproximación por mínimos cuadrados Monte Carlo” (LSM, por sus siglas en inglés) para evaluar opciones americanas con múltiples factores con el fin de estimar el “payoff” donde tradicionalmente no se podían utilizar diferencias finitas con el fin obtener rendimientos
	\end{abstract}

	\tableofcontents

	
	%%\begin{multicols}{2}

	\section{Introducción}
	Utilizaremos el técnica detallada en la investigación de Longstaff-Schwartz (2001, UCLA) denominada  “Aproximación por mínimos cuadrados Monte Carlo” (LSM, por sus siglas en inglés) para evaluar opciones americanas con múltiples factores con el fin de estimar el “payoff” donde tradicionalmente no se podían utilizar diferencias finitas.
	
	\subsection{Características}
	\begin{itemize}
		\item Es difícil estimar el “payoff” en situaciones donde la opción es afectada por más de 1 factor; esto es debido a que el método de diferencia finita y binomial son imprácticos con múltiples factores. Es de notarse que Wall Street utiliza diferencia finita sobre simplificando a 1 factor, aún cuando es demostrable que la opción es afectada por más de un factor.
		\item La simulación es una alternativa prometedora a diferencias finitas. 
		\item Esta técnica es útil para distintas opciones, entre las cuales se incluyen: hipotecas, forex, commodities, seguros, energía, swap, mercados emergentes, deuda soberana, convertibles.
		\item La simulación permite que las variables de estado sigan un proceso estocástico general, como “jump diffusions”.
		\item Funciona con semi martingalas
		\item Las simulaciones tienen las siguientes características: simples, paralelizables, transparentes y flexibles.
			\subitem Paralelizable: trabajan bien en ambientes computacionales paralelos: esto ayuda al tipo de escalabilidad que buscamos tener en el presente trabajo.
			\subitem Simpleza: el único método a implementar es el simple método de Mínimos Cuadrados.
		
		\item Este enfoque es intuitivo al ser determinado por la función de la esperanza condicional del payoff inmediato versus continuar ejerciendo la opción. Dicha condicional se puede estimar con el cruce seccional de información en la simulación utilizando mínimos cuadrados.
		Obteniendo la función de la esperanza condicional para cada fecha del ejercicio se podrá obtener la especificación óptima para ejercer cada trayectoria (“path”).
		
		En contraste investigaciones previas, desde Tilley (1993) hasta García (1999); no se utiliza Longstaff-Schwartz proponen no utilizar varias estratificaciones o técnicas de parametrización para aproximar la función de densidad de transición o acotar el ejercicio.
		\item Longstaff-Schwartz utiliza la estrategia de enfocarse directamente en la función de Esperanza Condicional. Otros artículos de investigación también han adoptado este enfoque. Sin embargo, Longstaff-Schwartz nos gustó por ser un enfoque más pragmático, y más eficiente computacionalmente. Esto es debido a que solo realiza regresiones sobre los paths que pueden ser monetizados en la opción. 
		\item Longstaff-Schwartz proponen un enfoque con buenos resultados de rendimiento y desempeño comparándolo con otras técnicas
		\item Nuestra principal referencia menciona cuatro artículos con distintos niveles de complejidad que muestran ilustrativamente diferentes escenarios de opciones que demuestran el método y su efectividad.
	\end{itemize}

	\section{Desarrollo}
	
	\subsection{Estimación de Parámetros}
	Sea $(S_t)_{\geq 0}$ un proceso estocástico que representa el precio de un activo, i.e. $S_t$ es el precio de la acción al tiempo $t$.
	
	Se dice que el proceso $(S_t)_{\geq 0}$ es un movimiento Browniano geométrico con parámetros $\mu,\sigma>0$ si es la solución de la ecuación diferencial estocástica
	$$dS_t = \mu S_t dt + \sigma S_t dW_t$$
	
	para algún valor inicial $S_0$.
	Esta expresión se puede reescribir como
	$$\frac{dS_t}{S_t} = \mu dt + \sigma dW_t$$
	
	Se puede dar una interpretación de la ecuación anterior de la siguiente manera
	$$\frac{S_{t+dt}-S_t}{S_t}=\frac{dS_t}{S_t} = \mu dt + \sigma dW_t = \mbox{contribución determinista + constribución estocástica}$$
	
	donde se supone que la contribución determinista es proporcional y la parte estocástica tiene ley Gaussiana.
	
	A la constante $\mu$ se le conoce como drift del proceso y $\sigma$ se conoce  como parámetro volatilidad o de difusión.
	
	Se puede demostrar que una solución explícita para la ecuación diferencial estocástica anterior es
	$$S_t =  S_0 \exp\left\{\alpha t + \sigma W_t\right\} =  S_0 \exp\left\{\mu t -\frac{1}{2}\sigma^2t + \sigma W_t\right\},$$
	donde $\alpha = \mu-\frac{1}{2}\sigma^2$.
	
	Observación:
	Si $\sigma = 0$, i.e. no hay ruido estocástico, entonces la ecuación se convierte en
	$$\frac{dS_t}{S_t} = \mu dt,$$
	equivalentemente
	$$\frac{d}{dt}\log(S_t) = \mu$$
	y por lo tanto
	$$S_t = S_0 e^{\mu t}$$
	
	Nótese que la diferencia entre la solución determinista y no determinista es el término $\sigma W_t-\frac{1}{2}\sigma^2t$.
	
	A partir de la solución $$S_t = S_0 \exp\left\{\mu t -\frac{1}{2}\sigma^2t + \sigma W_t\right\}$$
	se puede ver que $S_t$ es la exponencial de un movimiento Browniano, i.e. es la exponencial de una variable aleatoria con distribución normal. Equivalentemente $S_t$ tiene distribución log-normal. Esta es una de las razones por las que el movimiento Browniano es adecuado para aplicaciones financieras.
	
	Considérese $t_0=0 < t_1 < t_2 <\ldots<t_n=T$ puntos en el horizonte de tiempo $[0,T]$. Defínase $Y_1,\ldots,Y_n$ como
	
	$$Y_i = \frac{S_{t_i}-S_{t_{i-1}}}{S_{t_{i-1}}}=\frac{S_{t_i}}{S_{t_{i-1}}}-1$$
	
	Si se considera la expansión de Taylor de la función $\log(1+z)$, se tiene que que para $z$ suficientemente pequeña
	$$\log(1+z) = z-\frac{1}{2}z^2 + o(z) \approx z$$
	Entonces si $X_i = 1+ Y_i$, entonces
	$$\log(X_i) = \log(1+Y_i) \approx Y_i = \log(S_{t_i})-\log(S_{t_{i-1}})$$
	Esto significa que los rendimientos exactos son casi idénticos a los log-rendimientos
	$$X_i = \log(S_{t_i})-\log(S_{t_{i-1}}),\ i\in\{1,\ldots,n\}$$
	
	Sea $\Delta t = t_i - t_{i-1}$ para $i\in\{1,\ldots,n\}$. Generalmente $\Delta t = \frac{1}{252}$ si se consideran periodos anuales.
	
	$$X_i + \ldots + X_{i+n-1}=\sum_{k=i}^{i+n-1}[\log(S_{t_i})-\log(S_{t_{i-1}})]=\log(S_{i+n-1})-\log(S_{i-1})$$
	
	Finalmente
	$$X_i = \log(S_{t_i})-\log(S_{t_{i-1}}) = \log\left(\frac{S_{t_i}}{S_{t_{i-1}}}\right)$$
	$$=\alpha \Delta t + \sigma[W_{t_i}-W_{t_{i-1}}]\sim N(\alpha t , \sigma^2 t)$$
	Es decir,
	$$X_i = \log\left(\frac{S_{t_i}}{S_{t_{i-1}}}\right) = \alpha \Delta t + \sigma\sqrt{\Delta t}Z,$$
	donde $Z\sim N(0,1)$
	
	además, para $i\neq j$ $X_{t_i}$ es independiente de $X_{t_j}$ pues cada una depende de incrementos disjuntos del movimiento Browniano estándar.
	
	Estimación de parámetros
	
	Gracias al análisis anterior se puede considerar a $X_1,\ldots,X_n$ como variables aleatorias independientes, idénticamente distribuidas $N(\alpha\Delta t,\sigma^2t \Delta t)$. Los estimadores máximo verosímiles para $\alpha$ y $\sigma^2$ son
	$$\widehat{\alpha}=\frac{1}{\Delta t}\frac{1}{n}\sum_{i=1}^n X_i$$
	$$\widehat{\sigma}^2 = \frac{1}{\Delta t}\frac{1}{n-1}\sum_{i=1}^n(X_i-\bar{X})^2=\frac{\widehat{S}^2}{\Delta t}$$
	
	De aquí que un estimador para $\mu$ es
	$$\widehat{\mu} = \widehat{\alpha} + \frac{1}{2}\widehat{\sigma}^2$$
	
	Observación:
	Nótese que 
	$$\sum_{i=1}^n X_i = \sum_{i=1}^n [\log(S_{t_i})-\log(S_{t_{i-1}})]=\log(S_{t_n})-\log(S_0)$$
	
	De aquí que
	
	$$\widehat{\alpha}=\frac{1}{n\Delta t}[\log(S_{t_n})-\log(S_0)]$$
	
	y entonces
	
	$$\widehat{\mu} =\frac{\log(S_{t_n})-\log(S_0)}{n\Delta t}+ \frac{1}{\Delta t}\frac{1}{n-1}\sum_{i=1}^n(X_i-\bar{X})^2=\frac{\widehat{S}^2}{\Delta t}$$
	$$\widehat{\sigma}^2 = \frac{1}{\Delta t}\frac{1}{n-1}\sum_{i=1}^n(X_i-\bar{X})^2=\frac{\widehat{S}^2}{\Delta t}$$
	
	\subsection{Implementación de Paro}
	
	\textit{\href{https://www.dropbox.com/s/pfarddbjlv2tc5z/implementacion_paro.pdf?dl=0}{Ver cuaderno de R}}
	
	\subsection{Estimación de Parámetros}
	
	\textit{\href{https://www.dropbox.com/s/njszo4jprkzxhz6/estimacion_parametros_edp.pdf?dl=0}{Ver cuaderno de R}}
	

	\begin{thebibliography}{9}
			\bibitem{longstaff} Longstaff, F. \& Schwartz, E. (2001). Valuing American Options by Simulation: A Simple Least-Squares Approach. The Review of Financial Studies Spring 2001 Vol. IS. No. I, pp. 113-147 The Society for Financial Studies
	\end{thebibliography}
	
	%%\end{multicols}
	
	
\end{document}